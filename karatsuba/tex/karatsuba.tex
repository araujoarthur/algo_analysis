\documentclass[12pt,a4paper]{article}

% Font setup - Helvetica
\usepackage[scaled]{helvet}
\renewcommand\familydefault{\sfdefault} % Sans-serif as default

% Page geometry
\usepackage[a4paper, margin=1in]{geometry}

% Useful packages for computation
\usepackage{amsmath}    % Math environments
\usepackage{amsfonts}   % Math fonts
\usepackage{amssymb}    % Math symbols
\usepackage{graphicx}   % Images
\usepackage{float}      % Force figure/table placement
\usepackage{booktabs}   % Better tables
\usepackage{siunitx}    % Proper units and numbers
\usepackage{listings}   % Code listings
\usepackage{xcolor}     % Colors for code highlighting

% Listings configuration (optional)
\lstset{
  basicstyle=\ttfamily\small,
  keywordstyle=\color{blue},
  commentstyle=\color{gray},
  stringstyle=\color{orange},
  numbers=left,
  numberstyle=\tiny,
  stepnumber=1,
  numbersep=5pt,
  frame=single,
  breaklines=true
}

\setlength{\parindent}{2em} % Adjust '2em' as desired


% Title info
\title{The Karatsuba's Multiplication Algorithm}
\author{Arthur A. G. Santos}
\date{\today}

\begin{document}

\maketitle
\tableofcontents
\newpage

\section{Introduction}
There are many ways to realize the multiplication between two numbers. Most of us are used to (and probably only to) the algorithm that the third grade teacher taught us.
Indeed, that seems to be the easies way, but it's not the most efficient. In this text I aim to present the karatsuba's algorithm for multiplication.
The method is actually an optimization to the recursive multiplication algorithm, in terms that the recursive algorithm requires four multiplications, whilst karatsuba's exchange the cost of one multiplication
for the burden of a bunch of sums. 

\section{Methodology}
Explain computational methods, algorithms, formulas used.

\section{Results}
Present results, graphs, tables, etc.

\section{Discussion}
Interpret results, compare with expected outcomes.

\section{Conclusion}
Summarize findings, future work, etc.

\appendix
\section{Additional Data}
Any supplementary material, code snippets, detailed tables.

\end{document}
